%%%%%%%%%%%%%%%%%%%%%%%%%%%%%%%%%%%%%%%%%
% Beamer Presentation
% LaTeX Template
% Version 1.0 (10/11/12)
%
% This template has been downloaded from:
% http://www.LaTeXTemplates.com
%
% License:
% CC BY-NC-SA 3.0 (http://creativecommons.org/licenses/by-nc-sa/3.0/)
%
%%%%%%%%%%%%%%%%%%%%%%%%%%%%%%%%%%%%%%%%%

%----------------------------------------------------------------------------------------
%   PACKAGES AND THEMES
%----------------------------------------------------------------------------------------

\documentclass{beamer}

\mode<presentation> {

% The Beamer class comes with a number of default slide themes
% which change the colors and layouts of slides. Below this is a list
% of all the themes, uncomment each in turn to see what they look like.

%\usetheme{default}
%\usetheme{AnnArbor}
%\usetheme{Antibes}
%\usetheme{Bergen}
%\usetheme{Berkeley}
%\usetheme{Berlin}
%\usetheme{Boadilla}
%\usetheme{CambridgeUS}
%\usetheme{Copenhagen}
%\usetheme{Darmstadt}
%\usetheme{Dresden}
%\usetheme{Frankfurt}
%\usetheme{Goettingen}
%\usetheme{Hannover}
%\usetheme{Ilmenau}
%\usetheme{JuanLesPins}
%\usetheme{Luebeck}
\usetheme{Madrid}
%\usetheme{Malmoe}
%\usetheme{Marburg}
%\usetheme{Montpellier}
%\usetheme{PaloAlto}
%\usetheme{Pittsburgh}
%\usetheme{Rochester}
%\usetheme{Singapore}
%\usetheme{Szeged}
%\usetheme{Warsaw}

% As well as themes, the Beamer class has a number of color themes
% for any slide theme. Uncomment each of these in turn to see how it
% changes the colors of your current slide theme.

%\usecolortheme{albatross}
%\usecolortheme{beaver}
%\usecolortheme{beetle}
%\usecolortheme{crane}
%\usecolortheme{dolphin}
%\usecolortheme{dove}
%\usecolortheme{fly}
%\usecolortheme{lily}
%\usecolortheme{orchid}
%\usecolortheme{rose}
%\usecolortheme{seagull}
%\usecolortheme{seahorse}
%\usecolortheme{whale}
%\usecolortheme{wolverine}

%\setbeamertemplate{footline} % To remove the footer line in all slides uncomment this line
%\setbeamertemplate{footline}[page number] % To replace the footer line in all slides with a simple slide count uncomment this line

%\setbeamertemplate{navigation symbols}{} % To remove the navigation symbols from the bottom of all slides uncomment this line
}

\usepackage{graphicx} % Allows including images
\usepackage{booktabs} % Allows the use of \toprule, \midrule and \bottomrule in tables

%----------------------------------------------------------------------------------------
%   TITLE PAGE
%----------------------------------------------------------------------------------------

\title[Modern Software Development Practices]{Testing Scientific Codes} % The short title appears at the bottom of every slide, the full title is only on the title page

\author{Ryan Pepper} % Your name
\institute[University of Southampton] % Your institution as it will appear on the bottom of every slide, may be shorthand to save space
{
University of Southampton \\ % Your institution for the title page
\medskip
\textit{ryan.pepper@soton.ac.uk} % Your email address
}
\date{October 29th 2018} % Date, can be changed to a custom date

\begin{document}

\begin{frame}
\titlepage % Print the title page as the first slide
\end{frame}


\begin{frame}
\frametitle{Scientific Method}
Set of rough principles:
\begin{itemize}
\item Ask a question.
\item Form a hypothesis.
\item Make a prediction based on the hypothesis.
\item Design an experiment to test the hypothesis.
\item Build the equipment
\item \textbf{Calibrate equipment by testing it}
\item Run the experiment
\item Detail carefully the procedures and record information.
\item Draw conclusions from results.
\end{itemize}
\end{frame}

\begin{frame}
\frametitle{How often is this piece missing in scientific software?}
\begin{itemize}
    \item Often ad-hoc changes made to code which mean things stop working.
    \item If you don't run tests on the code, you will often come back and find previous versions have stopped working.
    \item Especially common when multiple people working on software.
\end{itemize}
\end{frame}

\begin{frame}
\frametitle{Principles of Software Testing}
\begin{itemize}
\item Software must meet the requirements
\item Software must respond correctly to different inputs
\item Software must run in a reasonable amount of time
\item Can be installed and run in the intended environment
\end{itemize}
\end{frame}

\begin{frame}
\frametitle{Static vs Dynamic Testing}
\begin{block}{Static Testing}
    \begin{itemize}
        \item Reviews - Reviewing the code manually as a group.
        \item Walthrough - Parties go through a products specification to give feedback on potential defects.
        \item \textbf{Verification of design}
    \end{itemize}
\end{block}

\begin{block}{Dynamic Testing}
    \begin{itemize}
        \item Executing portions of the code and checking the response.
        \item \textbf{Validation of design}
    \end{itemize}
\end{block}
\end{frame}

\begin{frame}
\frametitle{Why is it difficult for science?}
\begin{itemize}
    \item Much of the time you don't know what the results will be.
    \item Realistic problems can take time to run.
    \item 
\end{itemize}
\end{frame}

\begin{frame}
    \frametitle{Why is it crucial for science?}
    \begin{itemize}
        \item Often need to implement new algorithms or change things.
        \item Changing without testing will break things.
        \item You might need to come back to a previous experiment to use it as the basis for a new one; if the code is broken, you will have issues!
    \end{itemize}
\end{frame}

\begin{frame}
    \frametitle{Dynamic Testing: Types of Test}
    \begin{block}{Unit Tests}
        \begin{itemize}
            \item Code is written in small, independent functions
            \item Each function has tests that check that the result makes sense
            \item For e.g. a test of a function that calculates sin could check $\sin{\left(n\pi\right)} = 0 \,\,\,\,\, \forall n \in \mathbb{Z}$
        \end{itemize}
    \end{block}

    \begin{block}{Integration Tests}
        \begin{itemize}
            \item Tests check that the system as a whole works convincingly.
            \item For e.g. a test of a molecular dynamics code might test that the pressure is close to the expected value after some equilibrium time.
            \item Much, much harder to design, but essential - can be written so that they also serve as examples of how to use the software.
        \end{itemize}
    \end{block}

    \begin{block}{Regression Tests}
        \begin{itemize}
            \item Check that results don't get less accurate over time.
            \item Check that performance stays the same or gets better over time.
        \end{itemize}
    \end{block}
\end{frame}

\begin{frame}
    \frametitle{Things to watch out for...}
    \begin{itemize}
        \item Testing frameworks - pick one and stick with it.
        \item 
    \end{itemize}
\end{frame}

\begin{frame}
    \frametitle{Aside: Test Driven Development}
    Methodology for testing where \textbf{tests are written before writing any code}.

    This means...
    \begin{itemize}
    \item Before writing any code you need to plan out what the interface should be (i.e. what arguments get passed to functions).
    \item You must work out what the output should be before writing any code.
    \item Can form part of the 'agile' methodology for software development.
    \item Can be difficult for developing scientific code where answers can be difficult to determine in advance.
    \end{itemize}
\end{frame}

\begin{frame}
    \centering 
    A practical demonstration...
\end{frame}


\begin{frame}
    \frametitle{Continuous Integration}
    What is CI?
    \begin{itemize}
        \item Runs the tests repeatedly, every time you make a change.
        \item Sends you angry emails when tests fail!
        \item Usually done remotely, on a server.
        \item Requires you to be using version control.
        \item Can help when you're developing features in a branch/fork.
        \item Remote services: TravisCI, CircleCI, GitLab, 
        \item Self Hosted Services: Jenkins, Buildbot
    \end{itemize}
\end{frame}

\begin{frame}
    \frametitle{TravisCI}
    Pick this as it's easy to set up from what we've already done:
    \begin{itemize}
    \item Build Dockerfile from the repository
    \item Test software using a Docker container.
    \end{itemize}
\end{frame}
\end{document}